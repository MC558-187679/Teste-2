Modifique o pseudo-código do algoritmo de busca em profundidade apresentado em aula ou do CLRS (supondo que o grafo de entrada $G$ é orientado) para imprimir cada aresta $(u, v)$ juntamente com seu tipo (aresta da árvore, de avanço, de retorno ou de cruzamento). A complexidade do DFS modificado ainda dever ser $O(V + E)$.

Observação: basta escrever o pseudo-código, sem explicação ou prova de corretude; se você usou variáveis que não estavam no pseudo-código original de DFS, explique o que representam. Você deve escrever o pseudo-código inteiro.

\itemdsep

\newcommand{\Branco}{\const{Branco}\xspace}
\newcommand{\Cinza}{\const{Cinza}\xspace}
\newcommand{\Preto}{\const{Preto}\xspace}

\begin{codebox}
\Procname{$\proc{DFS}(G)$}
\li \Para \Cada $u \in V[G]$ \Faca
    \Do
\li     $cor[u] \Recebe \Branco$
\li     $\pi[u] \Recebe \Nulo$
    \End
\li $tempo \Recebe 0$
\li \Para \Cada $u \in V[G]$ \Faca
    \Do
\li     \Se $cor[u] = \Branco$
        \Do
\li         \Entao $\proc{DFS-Visit}(u)$
        \End
    \End
\end{codebox}

\begin{codebox}
\Procname{$\proc{DFS-Visit}(u)$}
\li $cor[u] \Recebe \Cinza$
\li $tempo \Recebe tempo + 1$
\li $d[u] \Recebe tempo$
\li \Para \Cada $v \in Adj[u]$
    \Do
\li     \Se $cor[v] = \Branco$ \Entao
        \Do
\li         \Entao
            \Do
\li             $\pi[v] \Recebe u$
\li             $\proc{DFS-Visit}(v)$
            \End
        \End
    \End
\li $cor[u] \Recebe \Preto$
\li $tempo \Recebe tempo + 1$
\li $f[u] \Recebe tempo$
\end{codebox}
